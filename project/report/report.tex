\documentclass[12pt, a4paper]{article}
\usepackage[utf8]{inputenc}
\usepackage{graphicx}
\usepackage{geometry}
\usepackage{float}
\usepackage{listings}
\usepackage{color}
\usepackage{hyperref}
\usepackage{amsmath}
\usepackage{subcaption}
\usepackage{fancyhdr}
\usepackage{times}

% Page layout
\geometry{left=2.54cm, right=2.54cm, top=2.54cm, bottom=2.54cm}

% Code listing style
\definecolor{codegreen}{rgb}{0,0.6,0}
\definecolor{codegray}{rgb}{0.5,0.5,0.5}
\definecolor{codepurple}{rgb}{0.58,0,0.82}
\definecolor{backcolour}{rgb}{0.95,0.95,0.92}

\lstdefinestyle{mystyle}{
    backgroundcolor=\color{backcolour},   
    commentstyle=\color{codegreen},
    keywordstyle=\color{magenta},
    numberstyle=\tiny\color{codegray},
    stringstyle=\color{codepurple},
    basicstyle=\ttfamily\footnotesize,
    breakatwhitespace=false,         
    breaklines=true,                 
    captionpos=b,                    
    keepspaces=true,                 
    numbers=left,                    
    numbersep=5pt,                  
    showspaces=false,                
    showstringspaces=false,
    showtabs=false,                  
    tabsize=2
}

\lstset{style=mystyle}

% Header setup
\pagestyle{fancy}
\fancyhf{}
\lhead{Student ID: 123090704}
\rhead{Name: Xu Zigeng}
\renewcommand{\headrulewidth}{0.4pt}

\begin{document}

\begin{titlepage}
    \thispagestyle{empty}
    \centering
    \vspace*{1cm}

    {\Large ECE3810 Microprocessor System Design \\ Laboratory \par}

    \vspace{3cm}

    {\Huge Laboratory Report 5 \par}

    \vspace{6cm}

    {\large
    Name: Xu Zigeng \\
    \vspace{0.8cm}
    Student ID: 123090704 \\
    \vspace{0.8cm}
    Date: 2025/12/15
    }

    \vfill

    {\Large The Chinese University of Hong Kong, Shenzhen \par}
    \vspace{2cm}
\end{titlepage}

\section{Objectives}
In this project, we aimed to design and build a 2-player bouncing ball game based on the knowledge learned through ECE3810 Labs 1-5. Specifically, the objectives were:
\begin{itemize}
    \item To implement a complete game system integrating GPIO, USART, LCD, External Interrupt, and Timer.
    \item To realize the basic bouncing ball game mechanics as specified in the project handout.
    \item To propose and implement improvements to make the game more challenging and engaging.
\end{itemize}

\section{Basics and Setup}
The system is built on the STM32F103 microprocessor. The hardware setup involves:
\begin{itemize}
    \item \textbf{STM32 Board}: The core controller managing game logic and peripherals.
    \item \textbf{TFT LCD}: Displays the game interface, including the ball, paddles, obstacles, and status text.
    \item \textbf{Joypad}: Connected via COM3, used by Player B to control the top paddle.
    \item \textbf{On-board Keys}: Key0, Key1, Key2 used by Player A for control and system settings.
    \item \textbf{USART}: Used for communication with the PC to receive a random seed for the ball's initial direction.
\end{itemize}

\begin{figure}[H]
    \centering
    \includegraphics[width=0.6\textwidth]{setup.jpg} 
    % \fbox{\parbox{10cm}{\vspace{5cm}\centering [Place Hardware Setup Photo Here]}}
    \caption{Hardware setup showing the STM32 board, LCD, and Joypad connection.}
    \label{fig:setup}
\end{figure}

\section{Experiment 1: Realization of Bouncing Ball Game}

\subsection{Initialization and Difficulty Selection}
Upon startup, the system initializes all peripherals. The LCD displays the welcome screen. Player A uses Key\_Up/Key1 or Player B uses the Joypad to select the difficulty level (Easy/Hard).

\begin{figure}[H]
    \centering
    \begin{subfigure}[b]{0.4\textwidth}
        \centering
        \includegraphics[width=0.8\textwidth]{welcome.jpg}
        % \fbox{\parbox{5cm}{\vspace{6cm}\centering [Place Welcome Screen Photo]}}
        \caption{Welcome screen with startup melody.}
        \label{fig:welcome_screen}
    \end{subfigure}
    \hfill
    \begin{subfigure}[b]{0.4\textwidth}
        \centering
        \includegraphics[width=0.8\textwidth]{Difficulty.jpg}
        % \fbox{\parbox{5cm}{\vspace{6cm}\centering [Place Difficulty Select Photo]}}
        \caption{Difficulty selection interface (Easy/Hard).}
        \label{fig:difficulty_select}
    \end{subfigure}
    \caption{Initial startup screen and difficulty selection.}
    \label{fig:welcome}
\end{figure}

\subsection{Random Direction via USART}
After difficulty selection, the game waits for a random seed. The Python script \texttt{ECE3080\_PC.exe} sends a random number (0-7) via USART. The system receives this seed to determine the ball's initial launch direction.

\begin{figure}[H]
    \centering
    \begin{subfigure}[b]{0.45\textwidth}
        \centering
        \includegraphics[width=\textwidth]{Waiting.jpg}
        % \fbox{\parbox{5cm}{\vspace{4cm}\centering [Place 'Waiting for Seed' Screen Photo]}}
        \caption{LCD showing waiting status.}
    \end{subfigure}
    \hfill
    \begin{subfigure}[b]{0.45\textwidth}
        \centering
        \includegraphics[width=\textwidth]{DisplaySeed.jpg}
        % \fbox{\parbox{5cm}{\vspace{4cm}\centering [Place PC GUI Photo]}}
        \caption{Showing the random seed.}
    \end{subfigure}
    \caption{USART communication process for random seed generation.}
    \label{fig:usart}
\end{figure}

\subsection{Countdown and Game Start}
Once the seed is received, a 3-second countdown is displayed. The ball then launches in the direction corresponding to the received random number.

\begin{figure}[H]
    \centering
    \includegraphics[width=0.45\textwidth]{CountDown.jpg}
    % \fbox{\parbox{8cm}{\vspace{5cm}\centering [Place Countdown Screen Photo Here]}}
    \caption{3-second countdown before the game starts.}
    \label{fig:countdown}
\end{figure}

\subsection{Gameplay and Physics}
During the game:
\begin{itemize}
    \item **Player A** controls the bottom paddle using Key2 (Left) and Key0 (Right).
    \item **Player B** controls the top paddle using the Joypad (Left/Right).
    \item The ball bounces off the side walls and paddles.
    \item A buzzer sounds upon collision.
    \item The elapsed time and bounce count are updated in real-time on the LCD.
\end{itemize}

\begin{figure}[H]
    \centering
    \includegraphics[width=0.45\textwidth]{Game.jpg}
    % \fbox{\parbox{8cm}{\vspace{5cm}\centering [Place Gameplay Screen Photo Here]}}
    \caption{Active gameplay showing the ball, paddles, and status information.}
    \label{fig:gameplay}
\end{figure}

\subsection{Game Over and Reset}
If a player fails to catch the ball, the game ends. The winner is announced, and the final stats are shown. After a few seconds, the game returns to the start screen.

\begin{figure}[H]
    \centering
    \includegraphics[width=0.45\textwidth]{Score.jpg}
    % \fbox{\parbox{8cm}{\vspace{5cm}\centering [Place Game Over Screen Photo Here]}}
    \caption{Game Over screen displaying the result and winner.}
    \label{fig:gameover}
\end{figure}

\section{Experiment 2: Improve the Bouncing Ball Game}

\subsection{Proposed Improvements}
To enhance the gameplay depth and user experience, I introduced several major improvements beyond the basic requirements:
\begin{enumerate}
    \item \textbf{Static Obstacles}: Added obstacles in the arena to create unpredictable bounce trajectories.
    \item \textbf{Dynamic Acceleration}: The ball's speed increases after every paddle hit, making the game progressively harder (rally intensity).
    \item \textbf{Best-of-Three Match System}: Implemented a scoring system where the first player to win 2 rounds wins the match, adding a competitive tournament feel.
    \item \textbf{Audio Feedback}: Added a welcome melody at startup and distinct sound effects for collisions (paddles vs. obstacles).
\end{enumerate}

\subsection{Realization}

\subsubsection{Static Obstacles}
Three obstacles were defined to disrupt simple reflection angles:
\begin{itemize}
    \item Center (Green): A rectangular block in the middle.
    \item Top Left (Magenta) \& Bottom Right (Cyan): Square blocks near the player zones.
\end{itemize}

\begin{lstlisting}[language=C, caption={Obstacle Definition}]
typedef struct {
    s16 x, y, w, h;
    u16 color;
} Obstacle;
// ... obstacles initialized in array ...
\end{lstlisting}

\subsubsection{Dynamic Acceleration}
To prevent infinite rallies, the ball's vertical velocity (\texttt{vy}) is incremented by a small factor upon every paddle collision, capped at a maximum limit.

\begin{lstlisting}[language=C, caption={Velocity Acceleration Logic}]
// Inside collision detection
if (abs(ball.vy) < MAX_SPEED) { 
    if (ball.vy > 0) ball.vy += ACCEL_STEP;
    else ball.vy -= ACCEL_STEP;
}
\end{lstlisting}

\subsubsection{Best-of-Three Logic}
Global counters track the number of rounds won by each player. The game loops until one player reaches 2 wins.

\begin{lstlisting}[language=C, caption={Match Winning Condition}]
if (winner == PLAYER_A) wins_a++;
else wins_b++;

if (wins_a >= 2 || wins_b >= 2) {
    current_state = MATCH_OVER; // End of match
} else {
    Reset_Round(); // Start next round
}
\end{lstlisting}

\subsubsection{Audio Effects}
A buzzer module is used to generate sound. A sequence of tones plays at the \texttt{WELCOME} state, and short beeps are triggered during collisions.

\subsection{Results}
The improvements significantly increased the game's engagement. The \textbf{acceleration} mechanic ensures rounds conclude eventually, while the \textbf{obstacles} force players to stay alert. The \textbf{sound effects} provide immediate feedback, and the \textbf{Best-of-Three} rule makes victory feel more earned.

\begin{figure}[H]
    \centering
    \begin{subfigure}[b]{0.4\textwidth}
        \centering
        \includegraphics[width=0.8\textwidth]{Game.jpg}
        % \fbox{\parbox{5cm}{\vspace{6cm}\centering [Place Obstacles Photo]}}
        \caption{Game with static obstacles (Green, Magenta, Cyan).}
        \label{fig:obstacles}
    \end{subfigure}
    \hfill
    \begin{subfigure}[b]{0.4\textwidth}
        \centering
        \includegraphics[width=0.8\textwidth]{BestOfThree.jpg}
        % \fbox{\parbox{5cm}{\vspace{6cm}\centering [Place Best-of-Three Photo]}}
        \caption{Best-of-Three match scoring system.}
        \label{fig:best_of_three}
    \end{subfigure}
    \caption{Improved game features: obstacles and match scoring.}
    \label{fig:improvement}
\end{figure}

\section{Conclusion}
This project successfully integrated various STM32 peripherals to create a functional two-player game. Experiment 1 demonstrated mastery of basic embedded system concepts (GPIO, Interrupts, Timers, USART, LCD). Experiment 2 further explored game logic and collision physics by introducing obstacles, resulting in a more engaging and challenging experience.

\end{document}

